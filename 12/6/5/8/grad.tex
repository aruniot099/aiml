\iffalse
\documentclass[journal,10pt,twocolumn]{article}
\usepackage{graphicx, float}
\usepackage[margin=0.5in]{geometry}
\usepackage{amsmath, bm}
\usepackage{array}
\usepackage{booktabs}
\usepackage{mathtools}

\providecommand{\norm}[1]{\left\lVert#1\right\rVert}
\let\vec\mathbf
\newcommand{\myvec}[1]{\ensuremath{\begin{pmatrix}#1\end{pmatrix}}}
\newcommand{\mydet}[1]{\ensuremath{\begin{vmatrix}#1\end{vmatrix}}}

\title{\textbf{Optimization Assignment}}
\author{Maddu Dinesh}
\date{September 2022}

\begin{document}

\maketitle
\paragraph{\textit{Problem Statement} -
\fi
At what points in the interval (0,2$\pi$) does the function $\sin2x$ attain its maximum value.
\\
\solution
	\begin{figure}[!ht]
		\centering
		\includegraphics[width=\columnwidth]{12/6/5/8/figs/a.png}
		\caption{}
		\label{fig:12/6/5/8}
  	\end{figure}
\iffalse
\section*{\large Figure}

\begin{figure}[H]
\centering
\includegraphics[width=1\columnwidth]{a.png}
\caption{Graph of f(x)}
\label{fig:triangle}
\end{figure}
\section*{\large Solution}

	
    \subsection*{\normalsize Gradient descent}
\fi    
  Since  
    \begin{align}
	\label{eq:12/6/5/8vol_varx}
	    f(x) &= \sin2x,
	    \\
	    f'(x) &= 2\cos2x
	\end{align}
\iffalse
we have to attain the maximum value of sin2x in the interval [0,2$\pi$]. This can be seen in Figure f(x).
\fi
Using gradient ascent, 
\begin{align}
	x_{n+1} &= x_n + \alpha \nabla f(x_n) \\
&=x_n+\alpha(2\cos2x)
\end{align}
Choosing
\begin{align}
	x_0&=0.5,\alpha=0.001, precision = 0.00000001, 
	\\
	f_{max} &= 1.0000,
 	x_{max}= 0.7854.
    \end{align}
   
    

    





 






